\section{Begriffserklärungen}

\subsection{Sterilisation}
Die Elimination (Abtrennung, Abtötung) aller Mikroorganismen, sowie die
Inaktivierung aller Viren, Plasmiden und DNS-Fragmenten, die sich in oder an einem Produkt oder Gegenstand befinden.\\
z.B. Autoklave, EO-Gas

\subsection{Desinfektion/Hygiene}
Die gezielte, partielle Verminderung der Keimzahl, vorzugsweise auf
Oberflächen (Keimzahlerniedrigung).
z.B. Händewaschen, Oberfläche mit Putzmittel reinigen

\subsection{Kontamination}
Produkte oder Gegenstände die nicht steril sind, werden als kontaminiert
bezeichnet.


